% Created 2016-12-07 mié 17:58
% Intended LaTeX compiler: pdflatex
\documentclass[11pt]{article}
\usepackage[utf8]{inputenc}
\usepackage[T1]{fontenc}
\usepackage{graphicx}
\usepackage{grffile}
\usepackage{longtable}
\usepackage{wrapfig}
\usepackage{rotating}
\usepackage[normalem]{ulem}
\usepackage{amsmath}
\usepackage{textcomp}
\usepackage{amssymb}
\usepackage{capt-of}
\usepackage{hyperref}
\author{Marc Corrales Berjano}
\date{\today}
\title{Example 2 A typical day at the computer}
\hypersetup{
 pdfauthor={Marc Corrales Berjano},
 pdftitle={Example 2 A typical day at the computer},
 pdfkeywords={},
 pdfsubject={},
 pdfcreator={Emacs 24.3.1 (Org mode 9.0.1)}, 
 pdflang={English}}
\begin{document}

\maketitle
\tableofcontents

\begin{itemize}
\item I will make an exaple for someone doing data analysis since I dont know which
\end{itemize}
are the advantages to 'real developping'.

Purpose :: We will take a look at the expression profiles of 25 Drosophila cell lines.

\section{{\bfseries\sffamily DONE} Start a new Docker container}
\label{sec:org2785f5f}
\section{{\bfseries\sffamily TODO} Start to track the project in git}
\label{sec:org133a7c0}
\begin{itemize}
\item Magit
\end{itemize}
\begin{verbatim}
git init 
git remote add origin https://github.com/histonemark/retreteando.git
git add *
git commit -m 'Retreteando'
\end{verbatim}

\section{{\bfseries\sffamily TODO} Get the data}
\label{sec:orgd258133}
\begin{itemize}
\item Lets download the data from GEO.
\end{itemize}
\begin{verbatim}
wget https://dl.dropboxusercontent.com/u/3975383/Drosophila_25_cell_lines.txt
ls | grep '^Drosophila'
\end{verbatim}

\section{{\bfseries\sffamily TODO} Clean and prepare the Data}
\label{sec:org1ff4c37}

\begin{itemize}
\item How does the data look like
\end{itemize}
\begin{verbatim}
head Drosophila_25_cell_lines.txt
\end{verbatim}

\begin{itemize}
\item Let's pretend that it was not so beautifully R ready and we needed to clean
\end{itemize}
it a bit and we have to remove 3 'evil' genes 'FBgn0000022','FBgn0000253' and 'FBgn0036608'.
\begin{verbatim}
import sys

toremove = ['FBgn0000022','FBgn0000253','FBgn0036608']

with open('Drosophila_25_cell_lines.txt') as f:
    sys.stdout.write('%s' % f.readline()) # Header
    for line in f:
        items = line.split()
        gname = items[0]
        if gname in toremove: continue
        sys.stdout.write('%s' % line)
\end{verbatim}
M-x toggle-truncate-lines
\section{{\bfseries\sffamily TODO} Explore the Data, Plot , statistics}
\label{sec:org72d158c}

\begin{itemize}
\item Lets load the clean data.
\end{itemize}
\begin{verbatim}
cells <- read.delim('./Dmel_25Cl_clean.tsv')
dim(cells)
\end{verbatim}

\begin{itemize}
\item Lets look at it
\end{itemize}
\begin{verbatim}
head(cells)
\end{verbatim}

\begin{itemize}
\item Lets see how the genes cluster.
\end{itemize}
\begin{verbatim}
#cells$color <- 'black'
#cells$color[cells$]
PCA <- prcomp(cells[7:29], scale=T) # Dont take RPKMs
plot(PCA$x)
\end{verbatim}


How pretty much all the genes seem to have a similar expression level except
for a group of 50-100 that seem to drive all the expression in the variance.
Blah blah blah.

\section{{\bfseries\sffamily TODO} Send a report, publish, export}
\label{sec:org5014627}
How do you want to save or show your code/results:
\begin{itemize}
\item As a pdf
\end{itemize}
\end{document}
